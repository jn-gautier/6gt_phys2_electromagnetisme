%importation des packages
\RequirePackage{pdfmanagement-testphase}
\DocumentMetadata {lang=fr}
\documentclass[fontsize=11pt,twoside=true,DIV=calc,parskip=half]{scrbook}
\usepackage{scrhack}
\usepackage{scrlayer-scrpage} %équivalent de fancyhdr pour kommascript
\usepackage[french]{babel}
\usepackage[autostyle]{csquotes} % csquotes va utiliser la langue définie dans babel
\usepackage[shortlabels]{enumitem} %permet de faire \begin{enumerate}[a)]
\usepackage{tabularx}
\usepackage{amsmath} %formules mathématiques
\usepackage{amssymb} %symboles mathématiques
\usepackage{amsfonts} %polices mathématiques
\usepackage{anyfontsize}
\usepackage{graphicx}
\usepackage{wrapfig}%image à doite ou a gauche du texte


\usepackage{multido}%pour la macro pointillés
\usepackage[breakable]{tcolorbox}
\tcbuselibrary{theorems}
\tcbuselibrary{skins,listings}
\usepackage{siunitx}
\sisetup{locale = FR}

\usepackage{pstricks}%pour dessiner les champs électriques
\usepackage{pst-magneticfield}%pour dessiner les champs magnétiques
\usepackage{pst-coil,pst-slpe,pstricks-add}

%hyperliens
\usepackage{hyperref}
\hypersetup{
  colorlinks,
  citecolor=black,
  filecolor=black,
  linkcolor=black,
  urlcolor=black
}


%police
%\usepackage{libertine} la police libertine et T1 ne permettent pas d'afficher les smallcaps, ni les upper+bold.
\usepackage[T1]{fontenc}


%cadre
%\usepackage{framed}

%graphique tikz
\usepackage{tikz}
\usepackage{tkz-base}
\usepackage{tkz-fct}
\usepackage{tkz-euclide}


%chemin vers les images
\graphicspath{ {./images/} }

%exercices xsim
\usepackage[clear-aux]{xsim}
\DeclareExerciseTagging{difficulty}
\xsimsetup{
  path=xsim,
  load-style = layouts ,
  exercise/template = runin ,
  solution/template = runin,
  difficulty={*,**,***},
}



%math
\everymath{\displaystyle} %pour que les équations soient bien présentées




%mise en page générale
\KOMAoptions{DIV=last}
\usepackage{setspace}%pour regler l'interligne
\onehalfspacing %interligne1.5
\setlength{\parindent}{0pt} %pas de retrait en début de paragraphe
\usepackage{caption}%mise en forme des légendes des figures
\addtokomafont{caption}{\scriptsize \slshape}
\addtokomafont{captionlabel}{\scriptsize \slshape}


%pieds de page utiliser scrlayer-scrpage
\setkomafont{pageheadfoot}{\small}
\lohead{} \rohead{} \cohead{}
\lofoot{Électromagnétisme - physique(2h) - 6GT}
\rofoot{Page \thepage}
\cofoot{}


%style perso et macro
\definecolor{NavyBlue}{HTML}{7279ff}
\definecolor{mygray1}{HTML}{e7e7e7}%light
\definecolor{mygray2}{HTML}{676767}%lightdark
\definecolor{Green}{HTML}{1d8600}%poleS
\definecolor{BrickRed}{HTML}{b30003}%poleN
\definecolor{Weiss}{rgb}{1,0.98,0.98}%  255 250 250 %main droite
\definecolor{Haut}{rgb}{1,0.894,0.769}% 255 228 196 %main droite
\definecolor{Auge}{rgb}{0.54,0.27,0.074}% 139 69 19 %main droite
\newcommand{\motcle}[1]{
  \uppercase{\textbf{#1}}
}

\newcommand{\pointilles}[1]{
  \begin{itemize}[label={}]
    \multido{}{#1}{\item \dotfill}
  \end{itemize}
}

\newenvironment{encadre}{\begin{center}\begin{minipage}[c]{0.8\linewidth}\begin{tcolorbox}[colframe=red!50!yellow,colback=red!5!white]}{\end{tcolorbox}\end{minipage}\end{center}}
\newtcbtheorem[number within=section]{encadre_equation}{}{colback=green!5,colframe=green!35!black,fonttitle=\bfseries}{th}


